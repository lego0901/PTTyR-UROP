
\documentclass{article}

\usepackage{pttyr_descriptions}


\begin{document}

\setlist{nolistsep}
\nointerlineskip
\par\noindent
\setlength{\parindent}{0pt}


\section*{Scalar Output}
\subsection*{\texttt{torch.numel}}
\prepost{torch.numel(input)}{
}{
  \begin{itemizec}
    \item $|\mtt{input}| = (d_1, d_2, \dots, d_k)$
  \end{itemizec}
}{
  \begin{itemizec}
    \item $d_1 \cdot d_2 \cdot \cdots \cdot d_k$를 스칼라로 반환
  \end{itemizec}
}
\begin{align*}
  \frac
  {
    \begin{array}{l}
      \sigma \vdash E \Rar e, c
    \end{array}
  }
  {
    \sigma \vdash \op{numel}{E} \Rar e[1] e[2] \cdots e[k], c
  }
\end{align*}

\subsection*{\texttt{torch.Tensor.dim}, \texttt{torch.Tensor.ndim},
\texttt{torch.Tensor.ndimension}}
\prepost{a.dim(), a.ndim or a.ndimension()}{
}{
  \begin{itemizec}
    \item $|\mtt{a}| = (d_1, d_2, \dots, d_k)$
  \end{itemizec}
}{
  \begin{itemizec}
    \item $k$를 스칼라로 반환
  \end{itemizec}
}
\begin{align*}
  \frac
  {
    \begin{array}{l}
      \sigma \vdash E \Rar e, c
    \end{array}
  }
  {
    \sigma \vdash E.\op{dim}{} \Rar \op{rank}{e}, c
  }
\end{align*}
\begin{align*}
  \frac
  {
    \begin{array}{l}
      \sigma \vdash E.\op{dim}{} \Rar k, c
    \end{array}
  }
  {
    \sigma \vdash E.\mtt{ndim} \Rar k, c
  }
  \tag*{alias}
\end{align*}
\begin{align*}
  \frac
  {
    \begin{array}{l}
      \sigma \vdash E.\op{dim}{} \Rar k, c
    \end{array}
  }
  {
    \sigma \vdash E.\mtt{ndimension()} \Rar k, c
  }
  \tag*{alias}
\end{align*}

\subsection*{\texttt{torch.eq}}
\prepost{torch.eq(input, other)}{
}{
}{
  \begin{itemizec}
    \item $input$과 $other$의 내용물과 shape이 모두 일치하면 $True$, 아니면 $False$
  \end{itemizec}
}
\begin{align*}
  \frac
  {
    \begin{array}{l}
      \sigma \vdash A \Rar \_, c_a \\
      \sigma \vdash B \Rar \_, c_b
    \end{array}
  }
  {
    \sigma \vdash torch.\op{eq}{A, B} \Rar bool, c_a \cup c_b
  }
\end{align*}


\section*{Conversion To/From NonTensor}
\subsection*{\texttt{torch.Tensor.tolist}}
\begin{itemize}
  \item 파이썬 리스트로 텐서를 바꿔주는 함수
\end{itemize}

\subsection*{\texttt{torch.as\_tensor(numpy\_list)}, \texttt{torch.from\_numpy}}
\begin{itemize}
  \item 넘파이 배열이나 파이썬 리스트로부터 텐서를 만드는 함수
\end{itemize}

\end{document}
